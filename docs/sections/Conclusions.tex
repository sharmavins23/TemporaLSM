\section{Conclusions} \label{sec:Conclusions}

Our project presents a framework for generating and analyzing self-organizing
neural networks in order to derive insight into the neural basis for cognition
through the orientation of the problem. Through analyzing the networks, we
uncovered interesting insights to the strength of inhibition in network design,
as well as the importance of the network's ability to be resilient in the face
of changing input data. Our work acts as a base to inspire particular design
choices for liquid state machines and temporal neural network design in general.

\subsection{Future Work} \label{sec:Future Work}

In future renditions of this work, we would like to explore a vast variety of
different network parameters and configurations, starting with the model of the
neuron. More complicated neuronal designs (discussed in
\ref{sec:Temporal Neuron}) may interact with STDP rules differently.

Other STDP models and models for Hebbian theory implementations may also be
explored. In particular, our project demonstrated to us the value and importance
of inhibition in biologically-plausible networks - As such, we are interested to
see if astrocyte modulation of synaptic weights or other forms of inhibition
would lead to more stable and accurate networks.

Finally, exploring the effects of differing networks on different datasets would
help draw similarities and differences between learning in the general sense.
Our structure is currently extensible to a variety of machine learning problems,
so picking relatively orthogonal problems could offer interesting insight to how
these problems may be solved in the brain.

\subsection{Contribution}

Throughout the semester, Vins focused on the implementation of the temporal
liquid state machine, while Anand focused on the construction and implementation
of the input vector and initial seed configuration networks. Both teammates
contributed to the overall design and analysis of the generated networks. Anand
focused on the information walk analysis, while Vins focused on analyzing the
network configurations and adjacency matrices.