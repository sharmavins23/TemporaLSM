\section{Approach} \label{sec:Approach}

\subsection{Dataset}

Throughout our project we apply the MNIST dataset \cite{MNIST Dataset}. This
dataset requires the tools of image recognition and classification, and is very
popular in generic machine learning projects.

As we are utilizing a controlled dataset for all models, any similarities or
differences in constructed graph topologies are fundamentally responses to the
individual parameter changes, the network structure, or the act of
'problem-solving' itself.

\subsection{Assumptions}

While we vary a variety of parameters, we assume that the underlying temporal
neurons function as a strong analogue for biological processes. This assumption
allows us to generalize our results to neural basis of cognition rather than
design of artificial neural networks.

\subsection{Analysis}

The focus of analysis is the network generated within the liquid reservoir. This
liquid has nodes of neurons and edges of weighted synapses.

We test the robustness claim of \cite{LSM Constraints} by analyzing the
connectivity of these networks over their growth. We also analyze diameter in
order to understand the 'distance' between neurons in the network (and the
amount of time it takes for a spike to propagate).

\subsection{Research Questions}

Through these network features, and more, we aim to understand the following:

\begin{enumerate}
    \item Why did the network configure itself the way it did?
    \item How do networks of different configurations compare to each other?
\end{enumerate}