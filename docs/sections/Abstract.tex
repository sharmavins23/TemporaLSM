\begin{abstract}
    Liquid state machines (LSMs) are a type of reservoir computer that aim to
    construct more biologically plausible neural network architectures than
    artificial counterparts. Rather than purposely constructing neurons in
    particular arrangements, an LSM self-organizes its neurons into a 'liquid',
    which is read out to perform a task. Many arguments against LSMs stem from
    this difficult-to-control random procedure, and due to their complicated
    setup, current LSM designs are not well explored.

    Through this work, we present TemporaLSM, a simulation framework for liquid
    state machines using good-for-hardware temporal neuron designs. Furthermore,
    we analyze a series of liquids generated by TemporaLSM to determine how
    different hyperparameters affect the construction of the liquid, and
    attempt to identify similarities between liquids on differing tasks.
\end{abstract}

\begin{IEEEkeywords}
    liquid state machines, temporal neural networks
\end{IEEEkeywords}