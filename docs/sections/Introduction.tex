\section{Introduction} \label{sec:Introduction}

\subsection{Motivation}

The portion of the brain primarily responsible for human thought and cognitive
function is thought to be the neocortex, which covers the outside shell. The
unfolded neocortex is the size of a dinner napkin, and is made up of tiny
perpendicular 'micro-fibers' called cortical columns \cite{Mountcastle}.

Cortical columns are thought to be the microcircuit that implements all
cognitive thought. Though it is well documented that different regions of the
brain are responsible for different functions, it has been shown \cite{Hawkins}
that the makeup of the cortical column is nearly identical across the entire
neocortex, excluding the inputs. This indicates that a circuit exists that can
be plugged into different problems and solve all of them without significant
modification or significant power consumption.

One approach to designing such a 'perfect' circuit is through induction. In
section \ref{sec:PreviousWork}, we'll discuss this approach in more detail, and
why we are not choosing it. However, another approach is from a top down - We
hope to generate networks that can solve a variety of problems, look at these
networks, and attempt to understand what similarities exist between them or how
different parameters and problem-specific features affect their design.

\subsection{General Approach}

We start by generating a framework to quickly and easily 'plug in' different
datasets and generate networks. These networks can organize themselves based on
features of the data, and we can analyze how they organize themselves over time.

Once we have a framework that can handle this, we aim to introduce a series of
unique problems and analyze how individual neurons in a network may connect.
Ideally, we can draw significant conclusions about the influence of various
parameters, and maybe even find similarities. These similarities and conclusions
can serve as the first steps in constructing a cortical column-like neural
circuit.